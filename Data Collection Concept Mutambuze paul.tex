\documentclass[options]{article}

 \usepackage[
    top    = 2.75cm,
    bottom = 2.50cm,
    left   = 4.00cm,
    right  = 3.50cm]{geometry}

\usepackage[parfill]{parskip}
\pagenumbering{roman}
\title{A SOLUTION TO LOGISTICAL CHALLENGES OF REMOTE BIOGAS PLANTS IN CENTRAL DISTRICTS OF UGANDA}
\author{Mutambuze Paul  16/U/7738/EVE  216012181\thanks{Lecturer: Dr. Ernest Mwebaze}}\newpage
\date{%
    Makerere University\\%
    Feb 22, 2018
}


\begin{document}
\begin{titlepage}
\maketitle
\end{titlepage}




\newpage
\pagenumbering{arabic} 
\section{\textbf{ Introduction}} 
Biogas typically refers to a gas produced by the biological breakdown of organic matter like animal dung 
in the absence of oxygen also known as anaerobic digestion or fermentation.
It can also be produced from other biodegradable materials such as sewage, 
garbage/refuse, plant materials and energy crops.
Biogas comprises primarily methane(60-70)percent and carbon dioxide(30-40)percent.


\subsection{\textbf{Background}}
Carbon Credits fund is emerging as one of the potential source of income for rural households that constructed 
biogas plants.
A fully functioning Biogas plant that is 3 (Three) years old and above benefits from the Carbon Credits fund.
This is a measure of
how much Methane the household has burnt through use of Biogas and prevented it from from damaging the Ozone layer.

\bigbreak

Considering the number of Biogas plants installed/constructed, SNV has categorised a total of 18 districts as 
remote hard to reach places. Alot of expenses are incurred to reach these areas.

\bigbreak
Its is for this reason therefore and more that Mutambuze Paul a computer Science student in
 second year came up with an online 
form to be filled in by both the new farmers acquiring the Biogas technology and those with the technology already.
This will simply require the farmer to fill it on their smart form and data will be compared with what already exists
in the database.


\subsection{\textbf{Problem Statement}}
SNV and TAALI have constructed over 2000 Biogas plants in the central region alone. Most of the beneficiaries of this technology
are rural farmers which poses a challenge of logistics in terms of getting information about the operation of the plants. Alot of 
expenses have been incurred to get these \textit{Maunal paper} forms filled and yet information required from the farmers can be inserted in an online form. 

\subsection{\textbf{Objectives}}


\subsubsection{\textbf{Main Objective}} 
The main objective of this report therefore is to automate the information on a paper form to make it fillable by the unlearned (illiterate) farmer.


\subsubsection{\textbf{Specific Objectives}}

\begin{itemize}
  \item To collect all the data necessary for completion of Form.
  \item To train farmers on how to fill the form.
  \item To perform a thorough analysis on the collected data.
  \item To come up with a conclusion from the data analysis.
\end{itemize}


\subsection{\textbf{Scope}}
The research is mainly for all potential farmers both in rural and peri-urban areas. Farmers who already have Biogas plants and those
without. Training of extension workers who will in turn train other farmers.

\subsection{\textbf{Research Significance}}
The aim of the research solves the problem of incurring expenses by SNV in getting information about the plant so as to make 
payments for carbon credits. Further more the research improves on the knowlgde base of the farmers by training them on how to 
use the new technology. We believe these and make this research significant for both the Organizations SNV and TAALI and also the 
common farmers.

\section{\textbf{Methodology}}
The proposed methodology consists of two phases, data collection and data analysis.\bigbreak
Data will be collected using ODK Collect, which will later on be uploaded to the ODK aggregate server to carry out all the required analysis.

\begin{itemize}
\item Enter Your Full Name: \textit{Mutambuze Paul}
\item Enter Your Phone Number: \textit{0775222529l}
\item District: \textit{Gomba}
\item Village: \textit{Jjagala}
\item Tap to Record GPS Location: \textit{Tap to record GPS Coordinates}
\item Map Location: \textit{Tap to view and record Map}
\item Plant Image: \textit{Take picture of Plant}
\item Mason Name: \textit{Ssenvewo}
\item Date of Completion: \textit{25/02/2018}
\item Biogas Construction Enterprise: \textit{Dropdown Menu}
\end{itemize}

\begin{thebibliography}{10} \bibitem{latexGuide} SNV, \emph{Logistical Challenges in Operations}, Available at \texttt{https://www.snv.org} \end{thebibliography}



\end{document}