
\documentclass[12pt]{article}
\usepackage{zed-csp,graphicx,color}%from

\begin{document}
\begin{figure}[h]
\centerline{\small MAKERERE 
\includegraphics[width=0.1\textwidth] {Muk_logo} UNIVERSITY}
\end{figure}
\centerline{COLLEGE OF COMPUTING AND INFORMATION SCIENCES\\}
\centerline{DEPARTMENT OF COMPUTER SCIENCE\\}
\centerline{COURSEWORK: Research Methodology(BIT 2207)\\}
\centerline{LECTURER: Dr. ERNEST MWEBAZE}
\paragraph{•}
\centerline{\begin{tabular}{|c|c|c|c|}
\hline
\textbf{No.}& \textbf{Student Name} & \textbf{RegNo} & \textbf{Std Number} \\ \hline
\textit{1}&\textbf{MUTAMBUZE PAUL} & \textit{16/U/7738/EVE}& \textit{216012181} \\ \hline 
 \hline
\end{tabular}}

\title{A SOLUTION TO LOCATION OF REMOTE BIOGAS PLANTS IN CENTRAL REGION}

\date{\today}
\maketitle




\paragraph{Biogas typically refers to a gas produced by the biological breakdown of organic matter like animal dung 
in the absence of oxygen also known as anaerobic digestion or fermentation.
It can also be produced from other biodegradable materials such as sewage, 
garbage/refuse, plant materials and energy crops.
Biogas comprises primarily methane(60-70) and carbon dioxide(30-40)percent. }


\begin{section}{Carbon Credit Fund Summary}\label{state} %labels allow us to refer to numbers latex generates.
Carbon Credits fund is emerging as one of the potential source of income for rural households that constructed 
biogas plants.
A fully functioning Biogas plant that is 3 (Three) years old and above benefits from the Carbon Credits fund.
This is a measure of
how much Methene the household has burnt through use of Biogas and prevented it from from damaging the Ozone layer.
Considering the number of Biogas plants installed/constructed, SNV has categorised a total of 18 districts as 
remote not easy to reach 
places. This is because beneficiaries are located deep in the villages making data collection a challenge.
This has in turn increased 
expenses incurred by the SNV team to move looking for household to household. 
Its is for this reason therefore and more that Mutambuze Paul a computer Science student in
 second year came up with an online 
form to be filled in by both the new farmers acquiring the technology and those with the technology already.
This will simply require the farmer to fill it on their smart form and data will be compared with what already exists
in the database. 

\end{section}

\begin{section}{Fields of the form:}\label{state} %labels allow us to refer to numbers latex generates.
\end{section}

\begin{tabular}{|c|c|c|}
\hline
\textbf{No.}& \textbf{Field Name} & \textbf{Sample input} \\ \hline
\textit{1}&\textbf{Enter Your Full Name} & \textit{Mutambuze Paul} \\ \hline 
\textit{2}&\textbf{Enter Your Phone Number} & \textit{0775222529} \\ \hline 
\textit{3}&\textbf{District} & \textit{Wakiso}\\ \hline 
\textit{4}&\textbf{Village} & \textit{Jjagala}\\ \hline 
\textit{5}&\textbf{Tap to Record GPS Location} & \textit{Tap to record GPS Coordinates}\\ \hline 
\textit{6}&\textbf{Map Location} & \textit{Tap to view and record Map} \\ \hline 
\textit{7}&\textbf{Plant Image} & \textit{Take Picture of Plant}\\ \hline 
\textit{8}&\textbf{Mason Name} & \textit{Ssenvewo}\\ \hline 
\textit{9}&\textbf{Date of Completion} & \textit{25/02/2018l} \\ \hline 
\textit{10}&\textbf{Biogas Construction Company} & \textit{Dropdown Menu}\\ 
 \hline
\end{tabular}
\paragraph{•}

\paragraph*{We believe with an automated data collection platform, this will remedy the challenges that there have been.
A farmer will simply tap Map Location to ascertain their exact location together with all other information at a click of a button.
This will eliminate expenses incurred in acquiring data from farmer.
 } 


\end{document}








